\documentclass[11pt,letterpaper]{article}

%\usepackage{fontspec}
%\usepackage[utf8]{inputenc}
\usepackage{textcomp,marvosym}
\usepackage{amsmath,amssymb}
\usepackage[normalem]{ulem}
\usepackage[left]{lineno}
\usepackage{booktabs}
\usepackage{changepage}
\usepackage{rotating}
\usepackage{color}
\usepackage{natbib}
\usepackage{setspace}
\usepackage{array}
\usepackage{fancyhdr}
\usepackage{graphicx}
\usepackage{xspace}
\usepackage[hidelinks]{hyperref}
\urlstyle{same}
\usepackage{threeparttable}
\doublespacing

\raggedright
\textwidth = 6.5 in
\textheight = 8.25 in
\oddsidemargin = 0.0 in
\evensidemargin = 0.0 in
\topmargin = 0.0 in
\headheight = 0.0 in
\headsep = 0.5 in
\parskip = 0.1 in
\parindent = 0.2in

% Bold the 'Figure #' in the caption and separate it from the title/caption with a period
% Captions will be left justified
\usepackage[aboveskip=1pt,labelfont=bf,labelsep=period,justification=raggedright,singlelinecheck=off]{caption}

% Remove brackets from numbering in List of References
%\makeatletter
%\renewcommand{\@biblabel}[1]{\quad#1.}
%\makeatother

% Self defined commands
\newcommand{\degC}{$^{\circ}$C\xspace}
\newcommand{\dC}{$\delta^{13}$C\xspace}
\newcommand{\dO}{$\delta^{18}$O\xspace}
\newcommand{\SrSr}{$^{87}$Sr/$^{86}$Sr\xspace}
\newcommand{\permil}{\textperthousand\xspace}
\newcommand{\dsil}{$d$\xspace}
\newcommand{\UPb}{$^{206}$Pb/$^{238}$U\xspace}
%

\pagestyle{myheadings}
\pagestyle{fancy}
\fancyhf{}
\lhead{Park et al., in preparation for XXX}
\rhead{\thepage}

\begin{document}

\begin{flushleft}
{\Large \textbf{Banxi Group}}

Yuem Park\textsuperscript{1},
Nicholas L. Swanson-Hysell\textsuperscript{1}

\bigskip
\textsuperscript{1} Department of Earth and Planetary Science, University of California, Berkeley, CA, USA

\textsuperscript{2} XXX
\bigskip

\end{flushleft}

\noindent\textit{This article is in preparation for XXX}

\linenumbers

\section*{ABSTRACT \label{sec:ABSTRACT}}

XXX

\section*{INTRODUCTION \label{sec:INTRODUCTION}}

XXX

\section*{METHODS}

XXX

\section*{RESULTS}

XXX

\section*{DISCUSSION}

\subsection*{South China Neoproterozoic APWP}

A potential complication with the pole from the Chengjiang Formation is that it is in a region on the western edge of the South China craton where there has been faulting and variable vertical-axis rotations associated with the Himalayan orogeny. \cite{Jing2019} point to the simalirty of nearby paleomagnetic directions from the Emeishan LIP to the overall province as supporting the area not having experienced such differential rotations. (we should take a look at this and see how that comparison actually looks)

The low-precision of dates obtained through SHRIMP complicates interpretation of the APWP. Given the spread in ages and the low precision of each analysis, it is hard to know if it should be young (ca. 780) with the older grains being inherited or if it should be old (ca. 820 Ma) with the younger grains having suffered Pb-loss. Whether a weighted mean of the entire population is warranted is uncertain.

XXX

Something that we can note is how the age constraints on the Banxi Group provide constraints on the collisional assembly of the Yangtze and Cathaysia blocks. Cawood2018 discuss how the younger bound on this assembly is provided by age of the Banxi given that they unconformably overlie the Sibao Group (an unconformity present where we worked). The age range that Cawood gives for the Banxi is 810 to 730 Ma with assembly of Yangtze and Cathaysia blocks extending to 810 Ma. Our preliminary dates appear to push that back a bit to ca. 815 Ma.

Reconstructions that have South China at moderately low latitudes ca. 800 Ma are: Pisaresky2003a

Li2008a has South China at high latitude ca. 810 Ma (based on Xiaofeng dikes) and then transiting to lower latitudes afterwards

\section*{ACKNOWLEDGEMENTS \label{sec:ACKNOWLEDGEMENTS}}

XXX

\clearpage
\newpage

\section*{TABLES}

XXX

\clearpage
\newpage

\section*{FIGURES}

XXX

\clearpage
\newpage
\footnotesize

\singlespacing

\bibliographystyle{gsabull}
\bibliography{References}

\end{document}
