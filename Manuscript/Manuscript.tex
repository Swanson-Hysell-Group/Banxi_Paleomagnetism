\documentclass[11pt,letterpaper]{article}

%\usepackage{fontspec}
%\usepackage[utf8]{inputenc}
\usepackage{textcomp,marvosym}
\usepackage{amsmath,amssymb}
\usepackage[normalem]{ulem}
\usepackage[left]{lineno}
\usepackage{booktabs}
\usepackage{changepage}
\usepackage{rotating}
\usepackage{color}
\usepackage{natbib}
\usepackage{setspace}
\usepackage{array}
\usepackage{fancyhdr}
\usepackage{graphicx}
\usepackage{xspace}
\usepackage[hidelinks]{hyperref}
\urlstyle{same}
\usepackage{threeparttable}
\doublespacing

\raggedright
\textwidth = 6.5 in
\textheight = 8.25 in
\oddsidemargin = 0.0 in
\evensidemargin = 0.0 in
\topmargin = 0.0 in
\headheight = 0.0 in
\headsep = 0.5 in
\parskip = 0.1 in
\parindent = 0.2in

% Bold the 'Figure #' in the caption and separate it from the title/caption with a period
% Captions will be left justified
\usepackage[aboveskip=1pt,labelfont=bf,labelsep=period,justification=raggedright,singlelinecheck=off]{caption}

% Remove brackets from numbering in List of References
%\makeatletter
%\renewcommand{\@biblabel}[1]{\quad#1.}
%\makeatother

% Self defined commands
\newcommand{\degC}{$^{\circ}$C\xspace}
\newcommand{\degrees}{$^{\circ}$\xspace}
\newcommand{\dC}{$\delta^{13}$C\xspace}
\newcommand{\dO}{$\delta^{18}$O\xspace}
\newcommand{\SrSr}{$^{87}$Sr/$^{86}$Sr\xspace}
\newcommand{\permil}{\textperthousand\xspace}
\newcommand{\dsil}{$d$\xspace}
\newcommand{\UPb}{$^{206}$Pb/$^{238}$U\xspace}
%

\pagestyle{myheadings}
\pagestyle{fancy}
\fancyhf{}
\lhead{Park et al., in preparation for XXX}
\rhead{\thepage}

\begin{document}

\begin{flushleft}
{\Large \textbf{Paleomagnetic and geochronologic data from the Banxi Group and the position of South China within the Supercontinent Rodinia during the early Neoproterozoic}}

Yuem Park\textsuperscript{1},
Nicholas L. Swanson-Hysell\textsuperscript{1}
Hanbiao Xian\textsuperscript{2},
Hairuo Fu\textsuperscript{3},
Daniel Condon\textsuperscript{4},
Shihong Zhang\textsuperscript{2},
Francis Macdonald\textsuperscript{5}

\bigskip
\textsuperscript{1} Department of Earth and Planetary Science, University of California, Berkeley, CA, USA

\textsuperscript{2} XXX
\bigskip

\end{flushleft}

\noindent\textit{This article is in preparation for XXX}

\linenumbers

\section*{ABSTRACT \label{sec:ABSTRACT}}

% from IGCP poster - needs updating
The late-Precambrian supercontinent Rodinia is hypothesized to have included almost all Proterozoic continental blocks. However, significant uncertainty regarding the configuration of these blocks as well as the timing of their assembly and breakup remains. Constraining this history is critical for understanding the tectonic boundary conditions during a critical interval in Earth’s history, the Neoproterozoic, which encompassed eukaryotic diversification, global glaciation (Snowball Earth), large carbon cycle fluctuations, large shifts in paleomagnetic poles interpreted to reflect inertial interchange true polar wander, and a putative rise in atmospheric oxygen to near-modern levels.

In particular, there is no consensus regarding the position of the S. China craton during this time, with competing models variably placing the craton at the core or on the periphery of Rodinia. High quality paleomagnetic data paired with precise geochronology from Neoproterozoic units in S. China can be used to distinguish between these models. This study presents paleomagnetic and geochronologic data from the Banxi Group in the Fanjingshan region of Guizhou province, China. The Banxi Group is a succession of siltstone and fine sandstone interbedded with volcanic ashes that are overlain by ca. 717--660~Ma Sturtian glacial deposits. Preliminary U-Pb CA-ID-TIMS zircon dates of ashes near the base of the Banxi Group demonstrate that deposition of the group in the study area began before ca. 815 Ma. Depending on the duration of deposition, this result implies that the Banxi Group has the potential to test the proposed Bitter Springs true polar wander events at ca. 810 and 790~Ma by yielding paleomagnetic data from before, between, and after the events. Paleomagnetic data reveal that Banxi Group siltstones preserve well-resolved high temperature components held by hematite. These high temperature components pass a regional fold test and are of dual polarity, both consistent with a primary magnetization. Together, the paleomagnetic and geochronologic data suggest that S. China was located at high paleolatitudes (\textgreater60\degrees) at ca. 815~Ma. We also redate a pole from the Liantuo Formation using U-Pb CA-ID-TIMS on zircon, which yields a preliminary date of ca. 778~Ma, significantly older than the previous U-Pb SHRIMP date of 748 $\pm$ 12~Ma. These results, as well as the inferred arc-affinity of igneous rocks along the western margin of the S. China craton and the post-Grenville age of terrane collisions associated with the Jiangnan Orogen, are difficult to reconcile with the missing-link model, which places S. China between Australia-East Antarctica and Laurentia. Furthermore, the orientation of S. China implied by the Banxi Group and Liantuo Formation poles are difficult to reconcile with southeastward terrane collisions onto a S. China that sits on the NW periphery of Rodinia ca. 778~Ma. Instead, they would place the craton at higher latitudes, likely in a different orientation on the periphery of Rodinia, or disconnected from it entirely.

\section*{INTRODUCTION \label{sec:INTRODUCTION}}

% FIGURES:
% regional map
% fence diagram

\section*{METHODS}

XXX

\section*{RESULTS}

% FIGURES:
% representative zijderveld plots
% representative site plots
%   geographic
%   tilt corrected
% site means
%   geographic dual polarity
%   tilt corrected dual polarity
%   tilt corrected single polarity
% bootstrap fold test

\section*{DISCUSSION}

\subsection*{South China Neoproterozoic APWP}

A potential complication with the pole from the Chengjiang Formation is that it is in a region on the western edge of the South China craton where there has been faulting and variable vertical-axis rotations associated with the Himalayan orogeny. \cite{Jing2019a} point to the similarity of nearby paleomagnetic directions from the Emeishan LIP to the overall province as supporting the area not having experienced such differential rotations. (we should take a look at this and see how that comparison actually looks)

The low-precision of dates obtained through SHRIMP complicates interpretation of the APWP. Given the spread in ages and the low precision of each analysis, it is hard to know if it should be young (ca. 780) with the older grains being inherited or if it should be old (ca. 820 Ma) with the younger grains having suffered Pb-loss. Whether a weighted mean of the entire population is warranted is uncertain.

XXX

Something that we can note is how the age constraints on the Banxi Group provide constraints on the collisional assembly of the Yangtze and Cathaysia blocks. Cawood2018 discuss how the younger bound on this assembly is provided by age of the Banxi given that they unconformably overlie the Sibao Group (an unconformity present where we worked). The age range that Cawood gives for the Banxi is 810 to 730 Ma with assembly of Yangtze and Cathaysia blocks extending to 810 Ma. Our preliminary dates appear to push that back a bit to ca. 815 Ma.

Reconstructions that have South China at moderately low latitudes ca. 800 Ma are: Pisaresky2003a

Li2008a has South China at high latitude ca. 810 Ma (based on Xiaofeng dikes) and then transiting to lower latitudes afterwards

\section*{ACKNOWLEDGEMENTS \label{sec:ACKNOWLEDGEMENTS}}

XXX

\clearpage
\newpage

\section*{TABLES}

XXX

\clearpage
\newpage

\section*{FIGURES}

XXX

\clearpage
\newpage
\footnotesize

\singlespacing

\bibliographystyle{gsabull}
\bibliography{References}

\end{document}
